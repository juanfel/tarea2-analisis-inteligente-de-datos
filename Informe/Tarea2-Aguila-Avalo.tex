\documentclass[letter, 10pt]{article}
\usepackage[utf8]{inputenc}
\usepackage[spanish]{babel}
\usepackage{amsfonts}
\usepackage{amsmath}
\usepackage[dvips]{graphicx}
\usepackage{graphicx}
\usepackage{subfigure} % subfiguras
\DeclareGraphicsExtensions{.bmp,.png,.pdf,.jpg}
\usepackage{xcolor,listings}%color support for listings
\usepackage{epstopdf}
\usepackage{algpseudocode}
\usepackage{algorithm}
\usepackage{url}
\usepackage{caption}
\usepackage{cite}
\usepackage[top=3cm,bottom=3cm,left=3.5cm,right=3.5cm,footskip=1.5cm,headheight=1.5cm,headsep=.5cm,textheight=3cm]{geometry}



\begin{document}



\title{Análisis Inteligente de Datos \\ \begin{Large}Tarea 2\end{Large}}
\author{Paulina Aguila - Juan Avalo}
\date{23 de junio de 2016}

\maketitle


\begin{figure}[ht]
\begin{center}
\includegraphics[width=0.2\textwidth]{Images/Isotipo-Negro.png}\\
\end{center}
\end{figure}
\vspace{2cm}
\section{Introducci\'on}
ACA INTRO

\section{Regresión Lineal Ordinaria (LSS)}

En esta sección, se estudiará un dataset llamado \textit{protate-cancer}\cite{datos}, que se utiliza a menudo con métodos de regresión. Los datos corresponden a un estudio realizado por Tom Stamey (Universidad de Stanford) en 1989 referente a la posible correlación entre el nivel de antígeno prostático específico (PSA) medido en un paciente, y otras mediciones clínicas que se obtuvieron luego de extirpar totalmente la próstata y los tejidos circundantes. Una de las variables que se estudian corresponden al volumen del cáncer prostático detectado enel paciente.


\subsection{Descripción de los datos}


\section{Selección de Atributos}

\section{Regularización}

\section{Predicción de Utilidades de Películas}

\section{Conclusiones}

ACA CONCLUSIONES

\bibliographystyle{plain}
\bibliography{Referencias}

\end{document} 